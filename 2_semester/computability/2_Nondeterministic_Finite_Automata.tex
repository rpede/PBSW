\documentclass{beamer}

\mode<presentation> {

\usetheme{Copenhagen}
\usecolortheme{seagull}
}

\usepackage{graphicx}
\DeclareGraphicsExtensions{.pdf,.png,.jpg}
\usepackage{booktabs} % Allows the use of \toprule, \midrule and \bottomrule in tables
\usepackage{mathtools}
\usepackage{multirow}
\usepackage{color, colortbl}
\usepackage{minted}

%----------------------------------------------------------------------------------------
%	TITLE PAGE
%----------------------------------------------------------------------------------------

\title[Test]{Test}

\author{Rasmus Guldborg Pedersen}

\date{June 2015}

\begin{document}

\begin{frame}
\titlepage
\end{frame}

\begin{frame}
\frametitle{Overview}
\tableofcontents
\end{frame}




\begin{frame}
    \frametitle{Nondeterministic Finite Automata}
    Define formally a non-deterministic finite automaton and the language
    accepted by a finite automaton. Describe a language over the alphabet $\{a,
    b\}$ that can be accepted by a finite automaton. Explain and justify your
    answer.
\end{frame}

\begin{frame}
    \frametitle{A Nondeterministic Finite Automaton}
    $(Q,\Sigma,q_0, A,\delta)$\\
    $Q$ is a finite set of \emph{states};\\
    $\Sigma$ is a finite \emph{input alphabet};\\
    $q_0 \in Q$ is the \emph{initial} state;\\
    $A \subseteq Q$ is the set of \emph{accepting} states;\\
    $\delta : Q \times (\Sigma \cup \{\Lambda\}) \rightarrow 2^Q$ is the
    \emph{transition} function.\\
    For $q \in Q$ and $\sigma \in (\Sigma \cup \{\Lambda\})$ then $\delta(q,
    \sigma)$ denotes the set of states the NFA can move to from $q$ on input
    $\sigma$.
\end{frame}

\begin{frame}
    \frametitle{Extended Transition Function $\delta^\ast$}
    $\delta^\ast : Q \times (\Sigma \cup \{\Lambda\}) \rightarrow 2^Q$\\
    $\delta^\ast (q, y\sigma) = \delta(\delta^\ast (q, y), \sigma)$
    \begin{enumerate}
        \item For every $q \in Q$, $\delta^\ast (q, y\sigma) =
            \Lambda(\{q\})$.\\
        \item For every $q \in Q$, every $y \in \Sigma^\ast$, and every $\sigma
            \in \Sigma$, $\delta^\ast (q, y\sigma) = \Lambda(\bigcup \{
            \delta(p, \sigma) | p \in \delta^\ast(q, y)\})$
    \end{enumerate}
\end{frame}

\begin{frame}
    \frametitle{Language accepted by a NFA}
    $L(M) = \{x \in \Sigma^\ast \left|\right (\delta^\ast(q_0, x)) \cap A \neq
    \emptyset\}$
\end{frame}

\begin{frame}
    \frametitle{Example}
    $L = \{a\} \cup \{b\}^\ast$\\

    \begin{tikzpicture}[>=stealth',shorten >=1pt,auto,node distance=2cm]
\node[initial,state]    (q0)                        {$q_0$};
\node[state]            (q1) [above right of=q0]    {$q_1$};
\node[state]            (q2) [right of=q1]          {$q_2$};
\node[state]            (q3) [below right of=q0]    {$q_3$};
\node[state,accepting]  (q4) [below right of=q2]    {$q_4$};

\path[->] (q0) edge              node {$\Lambda$}   (q1);
\path[->] (q1) edge              node {$a$}         (q2);
\path[->] (q2) edge              node {$\Lambda$}   (q4);
\path[->] (q0) edge              node {$\Lambda$}   (q3);
\path[->] (q3) edge [loop below] node {$b$}         (q3);
\path[->] (q3) edge              node {$\Lambda$}   (q4);
    \end{tikzpicture}
\end{frame}


%------------------------------------------------

\begin{frame}
    \frametitle{The End}

    %\Huge{\centerline{The End}}
    \begin{quote}
        ``Testing shows the presence, not the absence of bugs.''
        \raggedleft{--- Edsger W. Dijkstra}
    \end{quote}
\end{frame}

%----------------------------------------------------------------------------------------

\end{document}

