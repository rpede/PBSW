\documentclass{beamer}

\mode<presentation> {

\usetheme{Copenhagen}
\usecolortheme{seagull}
}

\usepackage{graphicx}
\DeclareGraphicsExtensions{.pdf,.png,.jpg}
\usepackage{booktabs} % Allows the use of \toprule, \midrule and \bottomrule in tables
\usepackage{mathtools}
\usepackage{multirow}
\usepackage{color, colortbl}
\usepackage{minted}

%----------------------------------------------------------------------------------------
%	TITLE PAGE
%----------------------------------------------------------------------------------------

\title[Test]{Test}

\author{Rasmus Guldborg Pedersen}

\date{June 2015}

\begin{document}

\begin{frame}
\titlepage
\end{frame}

\begin{frame}
\frametitle{Overview}
\tableofcontents
\end{frame}




\begin{frame}
    \frametitle{Deterministic Finite Automata}
    Define formally a finite automaton and the language accepted by a finite
    automaton. Describe a language over the alphabet $\{a, b\}$ that can be
    accepted by a finite automaton. Explain and justify your answer.
\end{frame}

\begin{frame}
    \frametitle{A Finite Automaton}
    $(Q,\Sigma,q_0, A,\delta)$\\
    $Q$ is a finite set of \emph{states};\\
    $\Sigma$ is a finite \emph{input alphabet};\\
    $q_0 \in Q$ is the \emph{initial} state;\\
    $A \subseteq Q$ is the set of \emph{accepting} states;\\
    $\delta : Q \times \Sigma \rightarrow Q$ is the \emph{transition}
    function.\\
    For $q \in Q$ and $\sigma \in \Sigma$ then $\delta(q, \sigma)$ denotes the
    state transition from $q$ on input $\sigma$.
\end{frame}

\begin{frame}
    \frametitle{Extended Transition Function $\delta^\ast$}
    $\delta^\ast : Q \times \Sigma^\ast \rightarrow Q$\\
    $\delta^\ast (q, y\sigma) = \delta(\delta^\ast (q, y), \sigma)$
\end{frame}

\begin{frame}
    \frametitle{Language accepted by a NFA}
    $L(M) = \{x \in \Sigma^\ast \left|\right (\delta^\ast (q_0, x)) \in A\}$
\end{frame}

\begin{frame}
    \frametitle{Example}
    The language over $\{a,b\}$ containing at least 1 $b$.

    \begin{tikzpicture}[>=stealth',shorten >=1pt,auto,node distance=2cm]
\node[initial,state]      (q0)               {$q_0$};
\node[state,accepting]    (q1) [right of=q0] {$q_1$};

\path[->] (q0) edge [loop above] node {a} (q0);
\path[->] (q0) edge              node {b} (q1);
\path[->] (q1) edge [loop above] node {a,b} (qq);
    \end{tikzpicture}
\end{frame}

\begin{frame}
    \frametitle{Example}
    $M = (Q,\Sigma,q_0, A,\delta)$\\
    $Q = \{q_0, q_1\}$;\\
    $\Sigma = \{a,b\}$;\\
    $A = \{q_1\}$ and $A \subseteq Q$;\\

    \vspace{10pt}
    $\delta$ is given by the table:\\

    \begin{tabular}{ c | c c }
        \hline
        $q$ & $\delta(q,a)$ & $\delta(q,b)$ \\
        \hline
        $q_0$ & $q_0$ & $q_1$ \\
        $q_1$ & $q_1$ & $q_1$ \\
        \hline
    \end{tabular}
\end{frame}


%------------------------------------------------

\begin{frame}
    \frametitle{The End}

    %\Huge{\centerline{The End}}
    \begin{quote}
        ``Testing shows the presence, not the absence of bugs.''
        \raggedleft{--- Edsger W. Dijkstra}
    \end{quote}
\end{frame}

%----------------------------------------------------------------------------------------

\end{document}

