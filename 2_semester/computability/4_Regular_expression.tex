\documentclass{beamer}

\mode<presentation> {

\usetheme{Copenhagen}
\usecolortheme{seagull}
}

\usepackage{graphicx}
\DeclareGraphicsExtensions{.pdf,.png,.jpg}
\usepackage{booktabs} % Allows the use of \toprule, \midrule and \bottomrule in tables
\usepackage{mathtools}
\usepackage{multirow}
\usepackage{color, colortbl}
\usepackage{minted}

%----------------------------------------------------------------------------------------
%	TITLE PAGE
%----------------------------------------------------------------------------------------

\title[Test]{Test}

\author{Rasmus Guldborg Pedersen}

\date{June 2015}

\begin{document}

\begin{frame}
\titlepage
\end{frame}

\begin{frame}
\frametitle{Overview}
\tableofcontents
\end{frame}




\begin{frame}
    \frametitle{Regular expression}
    Define formally a regular expression and the language generated by a
    regular language. Give an example of a regular expression and sketch how to
    construct a finite automaton accepting the same language as generated by
    the regular expression.
\end{frame}

\begin{frame}
    \frametitle{Regular Languages}
    If $\Sigma$ is an alphabet, then the set of regular languages is defined
    as:\\
    $\mathcal{R}$:
    \begin{enumerate}
        \item $\emptyset \in \mathcal{R}$
        \item For every $a \in \Sigma$, $\{a\} \in \mathcal{R}$
        \item For any $L_1$ and $L_2$ in $\mathcal{R}$,
            \[L_1 \cup L_2 \in \mathcal{R}\]
            \[L_1 L_2 \in \mathcal{R}\]
            \[L_1^\ast \in \mathcal{R}\]
    \end{enumerate}
\end{frame}

\begin{frame}
    \frametitle{Regular Expression}
    \begin{tabular}{ | l | l |}
        \hline
        \textbf{Regular Launguage}      &   \textbf{Regular Expression} \\
        \hline
        $\emptyset$                     &   $\emptyset$ \\
        $\{\Lambda\}$                   &   $\Lambda$ \\
        $\{a\}\{a\}$, $\{aa\}$          &   $aa$ \\
        $\{a\}\cup\{a\}$, $\{a, a\}$    &   $a + a$ \\
        $\{a\}^\ast$, ${a, aa, \dots}$  &   $a^\ast$ \\
        $\{a, b\}^\ast$                 &   $(a + b)^\ast$
    \end{tabular}
\end{frame}

\begin{frame}
    \frametitle{Language accepted by a FA}
    Let $M = (Q, \Sigma, q_0, A, \delta)$, and let $x \in \Sigma^\ast$.\\
    \[ L(M) = { x \in \Sigma^\ast | \delta^\ast(q_0, x) \in A } \]
    $M$ accepts $L$ if $L = L(M)$.
\end{frame}

\begin{frame}
    \frametitle{Examples}
    \begin{figure}
        \begin{tikzpicture}[>=stealth',shorten >=1pt,auto,node distance=2cm]
\node[initial,state]    (q0)                {$q_0$};
\node[state]            (q1) [right of=q0]  {$q_1$};
\node[state,accepting]  (q2) [right of=q1]  {$q_2$};

\path[->] (q0) edge node {$a$} (q1);
\path[->] (q1) edge node {$a$} (q2);
        \end{tikzpicture}
        \caption{$aa$}
    \end{figure}

    \begin{columns}
        \column{0.5\textwidth}
        \begin{figure}
            \begin{tikzpicture}[>=stealth',shorten >=1pt,auto,node distance=2cm]
\node[initial,state,accepting]    (q0)                        {$q_0$};

\path[->] (q0) edge [loop above]    node {$a$}  (q0);
            \end{tikzpicture}
            \caption{$a^*$}
        \end{figure}

        \column{0.5\textwidth}
        \begin{figure}
            \begin{tikzpicture}[>=stealth',shorten >=1pt,auto,node distance=2cm]
\node[initial,state]    (q0)                {$q_0$};
\node[state,accepting]  (q1) [right of=q0]  {$q_1$};

\path[->] (q0) edge [bend left]    node {$a$}  (q1);
\path[->] (q0) edge [bend right]    node {$b$}  (q1);
            \end{tikzpicture}
            \caption{$a+b$}
        \end{figure}
    \end{columns}
\end{frame}

\begin{frame}
    \frametitle{Examples}
    \begin{figure}
        \begin{tikzpicture}[>=stealth',shorten >=1pt,auto,node distance=2cm]
\node[initial,state]    (q0)                        {$q_0$};
\node[state]            (q1) [above right of=q0]    {$q_1$};
\node[state]            (q2) [right of=q1]          {$q_2$};
\node[state]            (q3) [below right of=q0]    {$q_3$};
\node[state,accepting]  (q4) [below right of=q2]    {$q_4$};

\path[->] (q0) edge              node {$\Lambda$}   (q1);
\path[->] (q1) edge              node {$a$}         (q2);
\path[->] (q2) edge              node {$\Lambda$}   (q4);
\path[->] (q0) edge              node {$\Lambda$}   (q3);
\path[->] (q3) edge [loop below] node {$b$}         (q3);
\path[->] (q3) edge              node {$\Lambda$}   (q4);
        \end{tikzpicture}
        \caption{$a + b^\ast$}
    \end{figure}
\end{frame}


%------------------------------------------------

\begin{frame}
    \frametitle{The End}

    %\Huge{\centerline{The End}}
    \begin{quote}
        ``Testing shows the presence, not the absence of bugs.''
        \raggedleft{--- Edsger W. Dijkstra}
    \end{quote}
\end{frame}

%----------------------------------------------------------------------------------------

\end{document}

