\documentclass{beamer}

\mode<presentation> {

\usetheme{Copenhagen}
\usecolortheme{seagull}
}

\usepackage{graphicx}
\DeclareGraphicsExtensions{.pdf,.png,.jpg}
\usepackage{booktabs} % Allows the use of \toprule, \midrule and \bottomrule in tables
\usepackage{mathtools}
\usepackage{multirow}
\usepackage{color, colortbl}
\usepackage{minted}

%----------------------------------------------------------------------------------------
%	TITLE PAGE
%----------------------------------------------------------------------------------------

\title[Test]{Test}

\author{Rasmus Guldborg Pedersen}

\date{June 2015}

\begin{document}

\begin{frame}
\titlepage
\end{frame}

\begin{frame}
\frametitle{Overview}
\tableofcontents
\end{frame}




\begin{frame}
    \frametitle{Non-deterministic Turing Machines as language acceptors}
    Give a definition of a non-deterministic Turing Machine. Describe how a
    non-deterministic Turing Machine processes its input and define the
    language accepted by a non-deterministic Turing machine.
\end{frame}

\begin{frame}
    \frametitle{Nondeterministic Turing Machine (informal)}
    \begin{tikzpicture}[>=stealth',shorten >=1pt,auto,node distance=1.6cm]
\node[initial,state]    (q0) {};
\node[state]            (q1) [right of=q0] {};
\node[state]            (q2) [right of=q1] {};
\node[state]            (q3) [right of=q2] {};
\node[state]            (q4) [right of=q3] {};
\node[state]            (q5) [right of=q4] {$h_a$};

\path[->] (q0) edge                 node {$\begin{array}{c}\Delta/\Delta\\R\end{array}$}    (q1);
\path[->] (q1) edge                 node {$\begin{array}{c}\Delta/1\\R\end{array}$}         (q2);
\path[->] (q2) edge                 node {$\begin{array}{c}\Delta/1\\R\end{array}$}         (q3);
\path[->] (q3) edge [loop below]    node {$\begin{array}{c}\Delta/1\\R\end{array}$}         (q3);
\path[->] (q3) edge                 node {$\begin{array}{c}\Delta/\Delta\\L\end{array}$}    (q4);
\path[->] (q4) edge [loop below]    node {$\begin{array}{c}1/1\\L\end{array}$}              (q4);
\path[->] (q4) edge                 node {$\begin{array}{c}\Delta/\Delta\\S\end{array}$}    (q5);
    \end{tikzpicture}
\end{frame}

\begin{frame}
    \frametitle{Nondeterministic Turing Machine (formal)}
    $T = (Q, \Sigma, \Gamma, q_0, \delta)$\\

    \vspace{10 pt}
    $Q$, a finite set of states\\
    $\Sigma$, the input alphabet ($\Sigma \subseteq \Gamma$)\\
    $\Gamma$, the tape alphabet ($\Delta \not\in \Gamma$)\\
    $q_0$, the initial state ($q_0 \in Q$)\\
    $\delta$, the transition function\\
    \[\delta \subseteq Q \times (\Gamma \cup \{\Delta\}) \rightarrow (Q \cup
    \{h_a, h_r\}) \times (\Gamma \cup \{\Delta\} \times \{R, L, S\}\]
\end{frame}

\begin{frame}
    \frametitle{Computational power}
    A NTM $T_1$ that acts as input to another $T_2$ can be thought of as
    executing $T_2$ in parallel branches.\\
    \pause
    This behavior can be simulated by a deterministic TM executing in a
    breadth-first manner.
\end{frame}

\begin{frame}
    \frametitle{Language accepted by a TM}
    If $x \in \Sigma$ then $x$ is accepted by $T$ if
    \[q_0 \Delta x \,\vdash^{\ast}_{T}\, w h_a y\]
    $L \subseteq \Sigma^\ast$ is accepted by $T$ if $L = L(T)$ where
    \[L(T) = \{x \in \Sigma^\ast \,|\, x \text{ is accepted by } T\}\]
\end{frame}


%------------------------------------------------

\begin{frame}
    \frametitle{The End}

    %\Huge{\centerline{The End}}
    \begin{quote}
        ``Testing shows the presence, not the absence of bugs.''
        \raggedleft{--- Edsger W. Dijkstra}
    \end{quote}
\end{frame}

%----------------------------------------------------------------------------------------

\end{document}

