\documentclass{beamer}

\mode<presentation> {

\usetheme{Copenhagen}
\usecolortheme{seagull}
}

\usepackage{graphicx}
\DeclareGraphicsExtensions{.pdf,.png,.jpg}
\usepackage{booktabs} % Allows the use of \toprule, \midrule and \bottomrule in tables
\usepackage{mathtools}
\usepackage{multirow}
\usepackage{color, colortbl}
\usepackage{minted}

%----------------------------------------------------------------------------------------
%	TITLE PAGE
%----------------------------------------------------------------------------------------

\title[Test]{Test}

\author{Rasmus Guldborg Pedersen}

\date{June 2015}

\begin{document}

\begin{frame}
\titlepage
\end{frame}

\begin{frame}
\frametitle{Overview}
\tableofcontents
\end{frame}




\begin{frame}
    \frametitle{Turing machines as a model universal computer}
    Give a formal definition of a Turing Machine. Describe the language
    accepted by a Turing machine and what it means that a Turing machine is
    total.
\end{frame}

\begin{frame}
    \frametitle{Turing Machine}
    $T = (Q, \Sigma, \Gamma, q_0, \delta)$\\

    \vspace{10 pt}
    $Q$, a finite set of states\\
    $\Sigma$, the input alphabet ($\Sigma \subseteq \Gamma$)\\
    $\Gamma$, the tape alphabet ($\Delta \not\in \Gamma$)\\
    $q_0$, the initial state ($q_0 \in Q$)\\
    $\delta$, the transition function\\
    \[\delta: Q \times (\Gamma \cup \{\Delta\}) \rightarrow (Q \cup \{h_a,
    h_r\}) \times (\Gamma \cup \{\Delta\} \times \{R, L, S\}\]
\end{frame}

\begin{frame}
    \frametitle{Language accepted by a TM}
    If $x \in \Sigma$ then $x$ is accepted by $T$ if
    \[q_0 \Delta x \,\vdash^{\ast}_{T}\, w h_a y\]
    $L \subseteq \Sigma^\ast$ is accepted by $T$ if $L = L(T)$ where
    \[L(T) = \{x \in \Sigma^\ast \,|\, x \text{ is accepted by } T\}\]
\end{frame}

\begin{frame}
    \frametitle{Total Turing Machine}
    A TM that always halts.\\
    \pause
    \vspace{10pt}
    Let $T$ be a TM that takes a description $T_1$ of a TM as input.\\
    $T$ \emph{accepts} if the $T_1$ halts, and \emph{rejects} otherwise.\\
    \pause
    $T'$ is an extended $T_1$ where if $T_1$ accepts it will loop forever, and
    if $T$ rejects it will accept.\\
    \pause
    Feed $T'$ to $T$.
\end{frame}


%------------------------------------------------

\begin{frame}
    \frametitle{The End}

    %\Huge{\centerline{The End}}
    \begin{quote}
        ``Testing shows the presence, not the absence of bugs.''
        \raggedleft{--- Edsger W. Dijkstra}
    \end{quote}
\end{frame}

%----------------------------------------------------------------------------------------

\end{document}

