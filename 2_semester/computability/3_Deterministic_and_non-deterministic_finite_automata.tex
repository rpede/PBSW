\documentclass{beamer}

\mode<presentation> {

\usetheme{Copenhagen}
\usecolortheme{seagull}
}

\usepackage{graphicx}
\DeclareGraphicsExtensions{.pdf,.png,.jpg}
\usepackage{booktabs} % Allows the use of \toprule, \midrule and \bottomrule in tables
\usepackage{mathtools}
\usepackage{multirow}
\usepackage{color, colortbl}
\usepackage{minted}

%----------------------------------------------------------------------------------------
%	TITLE PAGE
%----------------------------------------------------------------------------------------

\title[Test]{Test}

\author{Rasmus Guldborg Pedersen}

\date{June 2015}

\begin{document}

\begin{frame}
\titlepage
\end{frame}

\begin{frame}
\frametitle{Overview}
\tableofcontents
\end{frame}




\begin{frame}
    \frametitle{Deterministic and non-deterministic finite automata}
    Define the two types of finite automata. Are the classes of languages they
    define the same? Explain and justify your answer.
\end{frame}

\begin{frame}
    \frametitle{Finite Automata}
    $(Q,\Sigma,q_0, A,\delta)$\\
    $Q$ is a finite set of \emph{states};\\
    $\Sigma$ is a finite \emph{input alphabet};\\
    $q_0 \in Q$ is the \emph{initial} state;\\
    $A \subseteq Q$ is the set of \emph{accepting} states;\\
    $\delta$ is the \emph{transition} function.
\end{frame}

\begin{frame}
    \frametitle{Deterministic \& Nondeterministic FA}
    \begin{block}{DFA}
        $\delta : Q \times \Sigma \rightarrow Q$\\
        For $q \in Q$ and $\sigma \in \Sigma$ then $\delta(q, \sigma)$ denotes
        the state transition from $q$ on input $\sigma$.
    \end{block}

    \begin{block}{NFA}
        $\delta : Q \times (\Sigma \cup \{\Lambda\}) \rightarrow 2^Q$\\
        For $q \in Q$ and $\sigma \in (\Sigma \cup \{\Lambda\})$ then
        $\delta(q, \sigma)$ denotes the set of states the NFA can move to from
        $q$ on input $\sigma$.
    \end{block}
\end{frame}

\begin{frame}
    \frametitle{Eliminating $\Lambda$-transistions}
    $M = (Q,\Sigma,q_0,A,\delta)$\\
    $M_1 = (Q,\Sigma,q_0,A_1,\delta_1)$\\
    \pause
    $ A_1 = \left\{
        \begin{array}{l l}
            A \cup \{q_0\} & \quad \text{if $\Lambda \in L$ }\\
            A & \quad \text{if not}
    \end{array} \right.$\\
    \pause
    For every $q \in Q$ and every $\sigma \in \Sigma$ then $\delta_1(q,\sigma)
    = \delta^\ast(q,\sigma)$\\
\end{frame}

\begin{frame}
    \frametitle{DFA accepting the same language as an NFA}
    NFA $M = (Q, \Sigma, q_0, A, \delta)$\\
    FA $M_1 = (Q_1, \Sigma, q_1, A_1, \delta_1)$\\
    \pause
    $Q_1 = 2^Q$\\
    $q_1 = q_0$\\
    \pause
    $A_1 = \{q \in Q_1 | q \cap A \neq \emptyset \}$\\
    \pause
    $\delta_1(q,\sigma) = \bigcup\{ \delta(p,\sigma) | p \in q \}$\\
\end{frame}


%------------------------------------------------

\begin{frame}
    \frametitle{The End}

    %\Huge{\centerline{The End}}
    \begin{quote}
        ``Testing shows the presence, not the absence of bugs.''
        \raggedleft{--- Edsger W. Dijkstra}
    \end{quote}
\end{frame}

%----------------------------------------------------------------------------------------

\end{document}

