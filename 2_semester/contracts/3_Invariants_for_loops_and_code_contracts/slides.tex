\documentclass{beamer}

\mode<presentation> {

\usetheme{Copenhagen}
\usecolortheme{seagull}
}

\usepackage{graphicx}
\DeclareGraphicsExtensions{.pdf,.png,.jpg}
\usepackage{booktabs} % Allows the use of \toprule, \midrule and \bottomrule in tables
\usepackage{mathtools}
\usepackage{multirow}
\usepackage{color, colortbl}
\usepackage{minted}

%----------------------------------------------------------------------------------------
%	TITLE PAGE
%----------------------------------------------------------------------------------------

\title[Test]{Test}

\author{Rasmus Guldborg Pedersen}

\date{June 2015}

\begin{document}

\begin{frame}
\titlepage
\end{frame}

\begin{frame}
\frametitle{Overview}
\tableofcontents
\end{frame}



\section{Question}

\begin{frame}
    \frametitle{Invariants for loops and code contracts}

    What is an invariant for a loop, and how can it be used to reason about the
    behavior of a loop? Briefly explain what Code Contracts (.net tool) is and
    explain how it can be used to decorate a loop with contracts in order to
    ensure that a program assertion is an invariant.
\end{frame}

\section{Loop Invariants}

\begin{frame}[fragile]
    \frametitle{Loop Invariant}
    \begin{minted}[mathescape]{csharp}
    // $\{ Q \}$
    // $S_0$
    // $\{ P \}$
    while(B) {
        // $\{ P \wedge B \}$
        // $S$
        // $\{ P \}$
    }
    // $\{P \wedge \neg B \Rightarrow R \}$
    \end{minted}
\end{frame}

\begin{frame}
    \frametitle{Loop Invariant: Example}
    Algorithm for summing integers in a array.
    $a[0] + a[1] + \ldots a[N-1] = (\Sigma i | 0 \leq i < N : a[i])$
\end{frame}

\begin{frame}[fragile]
    \frametitle{Loop Invariant: Example}
    \begin{minted}[mathescape]{csharp}
// $\{ 0 \leq N \}$
int n = 0;
int s = 0;
// $\{ s = (\Sigma i | 0 \leq i < n : a[i]) \}$
while (n != N) {
    // $\{ s = (\Sigma i | 0 \leq i < n : a[i]) \wedge n \neq N \}$
    s = s + a[n];
    n = n + 1;
    // $\{ s = (\Sigma i | 0 \leq i < n : a[i]) \}$
}
// $\{ s = (\Sigma i | 0 \leq i < N : a[i]) \wedge n = N \}$
    \end{minted}
\end{frame}

\begin{frame}
    \frametitle{Loop Invariant: Example proof}
    \begin{block}{Basis: $n=1$}
        $a[0] = (\Sigma i | 0 \leq i < 1 : a[i])$
    \end{block}
    \begin{block}{Inductive step: $n+1$}
        $a[0] + a[1] + \ldots + a[n-1] + a[n] = (\Sigma i | 0 \leq i < n + 1 :
        a[i])$
    \end{block}
\end{frame}

\begin{frame}[fragile]
    \frametitle{Loop Invariant: Example proof}
    \begin{minted}[mathescape]{csharp}
while (n != N) {
    s = s + a[n];
    // $\{ s = (\Sigma i | 0 \leq i < n + 1 : a[i]) \}$
    n = n + 1;
}
    \end{minted}
\end{frame}

\begin{frame}
    \frametitle{Loop Invariant: Termination}
    Function $T$ such that loop execution ends when $T=0$.

    $T=N-n$ for the example.
\end{frame}


\section{Code Contracts}

\begin{frame}
    \frametitle{Code Contracts (.NET tool)}
    Express preconditions, postconditions and object invariants for:
    \begin{itemize}
        \pause \item Static analysis
        \pause \item Documentation
        \pause \item Runtime checking
    \end{itemize}
\end{frame}


\subsection{Example}

\begin{frame}
    \frametitle{Example using Code Contracts}
    Initialize and array $a$ with value $v$.
\end{frame}

\begin{frame}[fragile]
    \frametitle{Example: Basic Algorithm}
    \begin{minted}[mathescape]{csharp}
        int N = a.Length - 1;
        int n = 0;
        while (n != N) {
            a[n] = v;
            n = n + 1;
        }
    \end{minted}
\end{frame}

\begin{frame}[fragile]
    \frametitle{Example: Decorated}
    \begin{minted}[mathescape]{csharp}
Contract.Requires(a.Length > 0);
Contract.Ensures(Contract.ForAll(0, a.Length, 
    i => a[i] == v));
int N = a.Length;
int n = 0;
while (n != N) {
    a[n] = v;
    n = n + 1;
}
    \end{minted}
\end{frame}

\begin{frame}[fragile]
    \frametitle{Example: Decorated}
    \begin{minted}[mathescape]{csharp}
Contract.Requires(a.Length > 0);
Contract.Ensures(Contract.ForAll(0, a.Length,
    i => a[i] == v));
int N = a.Length;
int n = 0;
Contract.Assert(Contract.ForAll(0, n, i => a[i] == v));
while (n != N) {
    a[n] = v;
    n = n + 1;
    Contract.Assert(Contract.ForAll(0, n,
        i => a[i] == v));
}
    \end{minted}
\end{frame}

\begin{frame}
    \frametitle{Example: Termination}
    Termination function: $T=N-n$ \\
    When $n=N$ then $T=0$.
\end{frame}


%------------------------------------------------

\begin{frame}
    \frametitle{The End}

    %\Huge{\centerline{The End}}
    \begin{quote}
        ``Testing shows the presence, not the absence of bugs.''
        \raggedleft{--- Edsger W. Dijkstra}
    \end{quote}
\end{frame}

%----------------------------------------------------------------------------------------

\end{document}

