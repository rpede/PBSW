\documentclass{beamer}

\mode<presentation> {

\usetheme{Copenhagen}
\usecolortheme{seagull}
}

\usepackage{graphicx}
\DeclareGraphicsExtensions{.pdf,.png,.jpg}
\usepackage{booktabs} % Allows the use of \toprule, \midrule and \bottomrule in tables
\usepackage{mathtools}
\usepackage{multirow}
\usepackage{color, colortbl}
\usepackage{minted}

%----------------------------------------------------------------------------------------
%	TITLE PAGE
%----------------------------------------------------------------------------------------

\title[Test]{Test}

\author{Rasmus Guldborg Pedersen}

\date{June 2015}

\begin{document}

\begin{frame}
\titlepage
\end{frame}

\begin{frame}
\frametitle{Overview}
\tableofcontents
\end{frame}




\begin{frame}
    \frametitle{Pre- and post conditions for methods}

    What is pre- and post conditions for a method? You may give examples and
    show different levels of formality. How can program assertions be used to
    give a formal proof of the correctness of a method? How about termination,
    in case the implementation of the method contains a loop?
\end{frame}

\section{Pre- and Post-conditions}

\subsection{Assertions}

\begin{frame}
    \frametitle{Assertions}
    \begin{itemize}
        \item Assertion is a predicate.
        \pause \item A property we \emph{think} is true at that place during
            execution.
        \pause \item An assertion is valid if it's always true.
        \pause \item Abort if invalid.
    \end{itemize}
\end{frame}


\begin{frame}
    \frametitle{Preconditions}
    \begin{itemize}
        \item Evaluated before method execution.
        \pause \item Expectations from the caller.
        \pause \item Part of the specification.
    \end{itemize}
\end{frame}

\begin{frame}
    \frametitle{Postconditions}
    \begin{itemize}
        \item Evaluated after method execution.
        \pause \item What the caller can expect.
        \pause \item Part of the specification.
    \end{itemize}
\end{frame}


\subsection{Formalism}

\begin{frame}[fragile]
    \frametitle{Informal Specification}
    \begin{minted}{csharp}
public interface ISimpleDictionary {
    /* Put 'key' into the dictionary with associated
        'value'  */
    void Put(string key, object value);

    /* Remove 'key' from the dictionary */
    void Remove(string key);

    /* Does the dictionary contain 'key'? */
    bool ContainsKey(string key);
}
    \end{minted}
\end{frame}

\begin{frame}[fragile]
    \frametitle{Formal Specification}
    \begin{minted}{csharp}
public interface ISimpleDictionary {
    /* Pre: key != null && !ContainsKey(key)
        Post: ContainsKey(key) */
    void Put(string key, object value);

    /* Pre: key != null && ContainsKey(key)
        Post: !ContainsKey(key) */
    void Remove(string key);

    /* Pre: key != null */
    [Pure]
    bool ContainsKey(string key);
}
    \end{minted}
\end{frame}


\section{Proving Correctness}

\begin{frame}[fragile]
    \frametitle{Proving correctness}
    We can formalize specification of code with assertions.\\

    For an assertion to be valid we must prove correctness.\\

    Introducing loop invarients.
\end{frame}

\subsection{Using Loop Invariants}

\begin{frame}[fragile]
    \frametitle{Loop Invariant}
    \begin{minted}[mathescape]{csharp}
    // $\{ Q \}$
    // $S_0$
    // $\{ P \}$
    while(B) {
        // $\{ P \wedge B \}$
        // $S$
        // $\{ P \}$
    }
    // $\{P \wedge \neg B \Rightarrow R \}$
    \end{minted}
\end{frame}

\begin{frame}
    \frametitle{Loop Invariant: Example}
    Algorithm for summing integers in a array.
    $a[0] + a[1] + \ldots a[N-1] = (\Sigma i | 0 \leq i < N : a[i])$
\end{frame}

\begin{frame}[fragile]
    \frametitle{Loop Invariant: Example}
    \begin{minted}[mathescape]{csharp}
// $\{ 0 \leq N \}$
int n = 0;
int s = 0;
// $\{ s = (\Sigma i | 0 \leq i < n : a[i]) \}$
while (n != N) {
    // $\{ s = (\Sigma i | 0 \leq i < n : a[i]) \wedge n \neq N \}$
    s = s + a[n];
    n = n + 1;
    // $\{ s = (\Sigma i | 0 \leq i < n : a[i]) \}$
}
// $\{ s = (\Sigma i | 0 \leq i < N : a[i]) \wedge n = N \}$
    \end{minted}
\end{frame}

\begin{frame}
    \frametitle{Loop Invariant: Example proof}
    \begin{block}{Basis: $n=1$}
        $a[0] = (\Sigma i | 0 \leq i < 1 : a[i])$
    \end{block}
    \begin{block}{Inductive step: $n+1$}
        $a[0] + a[1] + \ldots + a[n-1] + a[n] = (\Sigma i | 0 \leq i < n + 1 :
        a[i])$
    \end{block}
\end{frame}

\begin{frame}[fragile]
    \frametitle{Loop Invariant: Example proof}
    \begin{minted}[mathescape]{csharp}
while (n != N) {
    s = s + a[n];
    // $\{ s = (\Sigma i | 0 \leq i < n + 1 : a[i]) \}$
    n = n + 1;
}
    \end{minted}
\end{frame}

\begin{frame}
    \frametitle{Loop Invariant: Termination}
    Function $T$ such that loop execution ends when $T=0$.

    $T=N-n$ for the example.
\end{frame}



%------------------------------------------------

\begin{frame}
    \frametitle{The End}

    %\Huge{\centerline{The End}}
    \begin{quote}
        ``Testing shows the presence, not the absence of bugs.''
        \raggedleft{--- Edsger W. Dijkstra}
    \end{quote}
\end{frame}

%----------------------------------------------------------------------------------------

\end{document}

