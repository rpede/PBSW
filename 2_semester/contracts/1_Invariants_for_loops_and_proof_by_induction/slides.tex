\documentclass{beamer}

\mode<presentation> {

\usetheme{Copenhagen}
\usecolortheme{seagull}
}

\usepackage{graphicx}
\DeclareGraphicsExtensions{.pdf,.png,.jpg}
\usepackage{booktabs} % Allows the use of \toprule, \midrule and \bottomrule in tables
\usepackage{mathtools}
\usepackage{multirow}
\usepackage{color, colortbl}
\usepackage{minted}

%----------------------------------------------------------------------------------------
%	TITLE PAGE
%----------------------------------------------------------------------------------------

\title[Test]{Test}

\author{Rasmus Guldborg Pedersen}

\date{June 2015}

\begin{document}

\begin{frame}
\titlepage
\end{frame}

\begin{frame}
\frametitle{Overview}
\tableofcontents
\end{frame}



\section{Question}

\begin{frame}
    \frametitle{Invariants for loops and proof by induction}

    What is an invariant for a loop, and how can it be used to give a formal
    proof of a loop? How can we argue that a loop will terminate? Explain proof
    by induction and relate it to how to prove that a program assertion is a
    loop invariant.
\end{frame}

\section{Proof by induction}

\begin{frame}
    \frametitle{Induction steps}
    Prove that $P(n)$ holds for all values of $n$. Where $n$ is a natural
    number.
    \begin{block}{Base case}
        Prove for some value of $n$ ($n=0$ or $n=1$).
    \end{block}
    \begin{block}{Inductive step}
        Prove for $n+1$.
    \end{block}
\end{frame}

\begin{frame}
    \frametitle{Example}
    $P(n): 0 + 1 + 2 + \cdots + n = \frac{n(n + 1)}{2}$
\end{frame}

\begin{frame}
    \frametitle{Example: Basis}
    $P(0): 0 = \frac{0(0 + 1)}{2}$
\end{frame}

\begin{frame}
    \frametitle{Example: Inductive step}
    Assume $P(n)$ holds. Show $P(n+1)$ holds.

    \begin{align}
    (0 + 1 + 2 + \cdots + n )+ (n+1) & = \frac{(n+1)((n+1) + 1)}{2} \\
    & = \frac{(n+1)(n+2)}{2} \\
    & = \frac{n(n+1)+2(n+1)}{2} \\
    & = \frac{n(n + 1)}{2} + (n+1)
    \end{align}
\end{frame}

\section{Loop Invariant}

\begin{frame}
    \frametitle{Invariant (general)}
    A predicate describing some property that can be relied upon always to be
    true.
\end{frame}

\begin{frame}[fragile]
    \frametitle{Loop Invariant}
    \begin{minted}[mathescape]{csharp}
    // $\{ Q \}$
    // $S_0$
    // $\{ P \}$
    while(B) {
        // $\{ P \wedge B \}$
        // $S$
        // $\{ P \}$
    }
    // $\{P \wedge \neg B \Rightarrow R \}$
    \end{minted}
\end{frame}

\begin{frame}
    \frametitle{Loop Invariant: Example}
    Algorithm for summing integers in a array.
    $a[0] + a[1] + \ldots a[N-1] = (\Sigma i | 0 \leq i < N : a[i])$
\end{frame}

\begin{frame}[fragile]
    \frametitle{Loop Invariant: Example}
    \begin{minted}[mathescape]{csharp}
// $\{ 0 \leq N \}$
int n = 0;
int s = 0;
// $\{ s = (\Sigma i | 0 \leq i < n : a[i]) \}$
while (n != N) {
    // $\{ s = (\Sigma i | 0 \leq i < n : a[i]) \wedge n \neq N \}$
    s = s + a[n];
    n = n + 1;
    // $\{ s = (\Sigma i | 0 \leq i < n : a[i]) \}$
}
// $\{ s = (\Sigma i | 0 \leq i < N : a[i]) \wedge n = N \}$
    \end{minted}
\end{frame}

\begin{frame}
    \frametitle{Loop Invariant: Example proof}
    \begin{block}{Basis: $n=1$}
        $a[0] = (\Sigma i | 0 \leq i < 1 : a[i])$
    \end{block}
    \begin{block}{Inductive step: $n+1$}
        $a[0] + a[1] + \ldots + a[n-1] + a[n] = (\Sigma i | 0 \leq i < n + 1 :
        a[i])$
    \end{block}
\end{frame}

\begin{frame}[fragile]
    \frametitle{Loop Invariant: Example proof}
    \begin{minted}[mathescape]{csharp}
while (n != N) {
    s = s + a[n];
    // $\{ s = (\Sigma i | 0 \leq i < n + 1 : a[i]) \}$
    n = n + 1;
}
    \end{minted}
\end{frame}

\begin{frame}
    \frametitle{Loop Invariant: Termination}
    Function $T$ such that loop execution ends when $T=0$.

    $T=N-n$ for the example.
\end{frame}



%------------------------------------------------

\begin{frame}
    \frametitle{The End}

    %\Huge{\centerline{The End}}
    \begin{quote}
        ``Testing shows the presence, not the absence of bugs.''
        \raggedleft{--- Edsger W. Dijkstra}
    \end{quote}
\end{frame}

%----------------------------------------------------------------------------------------

\end{document}

