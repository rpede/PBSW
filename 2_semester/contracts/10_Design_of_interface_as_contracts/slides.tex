\documentclass{beamer}

\mode<presentation> {

\usetheme{Copenhagen}
\usecolortheme{seagull}
}

\usepackage{graphicx}
\DeclareGraphicsExtensions{.pdf,.png,.jpg}
\usepackage{booktabs} % Allows the use of \toprule, \midrule and \bottomrule in tables
\usepackage{mathtools}
\usepackage{multirow}
\usepackage{color, colortbl}
\usepackage{minted}

%----------------------------------------------------------------------------------------
%	TITLE PAGE
%----------------------------------------------------------------------------------------

\title[Test]{Test}

\author{Rasmus Guldborg Pedersen}

\date{June 2015}

\begin{document}

\begin{frame}
\titlepage
\end{frame}

\begin{frame}
\frametitle{Overview}
\tableofcontents
\end{frame}




\begin{frame}
    \frametitle{Design of interfaces as contracts}

    Present a number of principles for designing interfaces as contracts. You
    may give examples in order to clarify the principles. How can we specify
    interfaces? You may include Code Contracts.
\end{frame}

\section{The 6 Principles}

\begin{frame}
    \frametitle{Principle 1}
    \begin{block}{Separate queries from commands.}
        Example: \emph{Pop} for a stack.\\
        \pause Introduce \emph{Peek} and \emph{Remove}.
        % Queries return result but do not change the visible properties
        % of the object.\\
        % Commands might change object but do not return a result.
    \end{block}
\end{frame}

\begin{frame}
    \frametitle{Principle 2}
    \begin{block}{Separate basic queries from derived queries.}
        Example: \emph{Peek} is derived from \emph{ElementAt}.
        %Derived queries can be specified in terms of basic queries.
    \end{block}
\end{frame}

\begin{frame}
    \frametitle{Principle 3}
    \begin{block}{For each derived query, write a postcondition that specifies
        what result will be returned in terms of one or more basic queries.}
        Assume: \emph{Count}\\
        \pause $Peek \to Post: Result = ElementaAt(Count)$
        %Then, if we know the values of the basic queries, we also know the
        %values of the derived queries.
    \end{block}
\end{frame}

\begin{frame}
    \frametitle{Principle 4}
    \begin{block}{For each command, write a postcondition that specifies the
        value of every basic query.}
        $Remove \to Post: Count = \textbf{old} Count - 1$
        %Now we know the total visible effect of each command.
    \end{block}
\end{frame}

\begin{frame}
    \frametitle{Principle 5}
    \begin{block}{For every query and command, decide on a suitable
        precondition.}
        $Peek \to Pre: Count > 0$
        %Preconditions constrain when clients may call the queries and commands.
    \end{block}
\end{frame}

\begin{frame}
    \frametitle{Principle 6}
    \begin{block}{Write invariants to define unchanging properties of objects.}
        $Invariant: Count \geq 0$
        %Concentrate on properties that help the reader build an appropriate
        %conceptual model of the abstraction that the class embodies.
    \end{block}
\end{frame}

\section{Interface Specification}

\begin{frame}
    \frametitle{Interface Specification}
    Interface specifications using Code Contracts.
\end{frame}

\begin{frame}[fragile]
    \frametitle{Interface}

    \begin{minted}{csharp}
public interface ISimpleQueue {
    void Enqueue(object item);
    object Dequeue();
    object ElementAt(int index);
    int Count();
}
    \end{minted}
\end{frame}

\begin{frame}[fragile]
    \frametitle{Contract}

    \begin{minted}{csharp}
abstract class ISimpleQueueContract {
    public void Enqueue(object item) {
        Contract.Requires(item != null);
        Contract.Ensures(Count ==
            Contract.OldValue(Count()) + 1);
        Contract.Ensures(ElementAt(Count()) == item);
        // ...
    }
    // ...
}
    \end{minted}
\end{frame}

\begin{frame}[fragile]
    \frametitle{Associating Interface with Contract}

    \begin{minted}{csharp}
[ContractClass(typeof(ISimpleQueueContract))]
public interface ISimpleQueue { /* ... */ }

[ContractClassFor(typeof(ISimpleQueue))]
abstract class ISimpleQueueContract { /* ... */ }
    \end{minted}
\end{frame}



%------------------------------------------------

\begin{frame}
    \frametitle{The End}

    %\Huge{\centerline{The End}}
    \begin{quote}
        ``Testing shows the presence, not the absence of bugs.''
        \raggedleft{--- Edsger W. Dijkstra}
    \end{quote}
\end{frame}

%----------------------------------------------------------------------------------------

\end{document}

