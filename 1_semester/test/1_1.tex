\documentclass{beamer}

\mode<presentation> {

\usetheme{Copenhagen}
\usecolortheme{seagull}
}

\usepackage{graphicx}
\DeclareGraphicsExtensions{.pdf,.png,.jpg}
\usepackage{booktabs} % Allows the use of \toprule, \midrule and \bottomrule in tables
\usepackage{mathtools}
\usepackage{multirow}
\usepackage{color, colortbl}
\usepackage{minted}

%----------------------------------------------------------------------------------------
%	TITLE PAGE
%----------------------------------------------------------------------------------------

\title[Test]{Test}

\author{Rasmus Guldborg Pedersen}

\date{June 2015}

\begin{document}

\begin{frame}
\titlepage
\end{frame}

\begin{frame}
\frametitle{Overview}
\tableofcontents
\end{frame}




\section{Q 1.1: Risk management in testing}

\subsection{Theory}
\begin{frame}
    \frametitle{Definition}
    \textbf{Risk}\\
    Anything that might prevent the project from delivering the right product
    on time and on budget.
\end{frame}

\begin{frame}
    \frametitle{Risk management}
    \begin{itemize}
        \item Identification
            % Identifying different risk items (project & quality).
        \item Analysis
            % Assessing the level of risk for each risk item. See ISO 9126.
        \item Mitigation/control
            % Mitigation contingency transference and acceptance actions for
            % each risk. Implementing protective measures for reducing/managing
            % the risks.
    \end{itemize}
\end{frame}

\begin{frame}
    \frametitle{Risk Types}
    \begin{itemize}
        \item Product/quality risks
        \item Project/planning risks
    \end{itemize}
\end{frame}

\begin{frame}
    \frametitle{Risk levels}
    \begin{itemize}
        \item Probability
        \item Impact
    \end{itemize}
\end{frame}

\subsection{Examples}
\begin{frame}
    \frametitle{Examples}
    We are testing a coffee machine.
    \resizebox{\columnwidth}{!}{%
    \begin{tabular}{ | l | l | l | }
        \hline
        \textbf{Risk} & \textbf{Mitigation} & \textbf{Contingency} \\ \hline
        Heater stops working & Reliability test the heater & Write ``out
        of order'' on display \\ \hline
        Doesn't meet the target price & Include headroom. & Use cheaper
        components \\ \hline
        Falling behind schedule & Monitor progress & Simplify the design \\
        \hline
    \end{tabular}%
    }
\end{frame}


%------------------------------------------------

\begin{frame}
    \frametitle{The End}

    %\Huge{\centerline{The End}}
    \begin{quote}
        ``Testing shows the presence, not the absence of bugs.''
        \raggedleft{--- Edsger W. Dijkstra}
    \end{quote}
\end{frame}

%----------------------------------------------------------------------------------------

\end{document}

