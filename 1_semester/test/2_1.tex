\documentclass{beamer}

\mode<presentation> {

\usetheme{Copenhagen}
\usecolortheme{seagull}
}

\usepackage{graphicx}
\DeclareGraphicsExtensions{.pdf,.png,.jpg}
\usepackage{booktabs} % Allows the use of \toprule, \midrule and \bottomrule in tables
\usepackage{mathtools}
\usepackage{multirow}
\usepackage{color, colortbl}
\usepackage{minted}

%----------------------------------------------------------------------------------------
%	TITLE PAGE
%----------------------------------------------------------------------------------------

\title[Test]{Test}

\author{Rasmus Guldborg Pedersen}

\date{June 2015}

\begin{document}

\begin{frame}
\titlepage
\end{frame}

\begin{frame}
\frametitle{Overview}
\tableofcontents
\end{frame}



\section{Q 2.1: Equivalence partitioning}

\begin{frame}
    \frametitle{Specification-Based Test Techniques}
    \begin{itemize}
        \item Black-box testing
        \item Test cases derived from specification
        \item Test the system according to the specification
    \end{itemize}
\end{frame}

\begin{frame}
    \frametitle{Equivalence Partitioning}
    \begin{itemize}
        \item Program as a function $P(x)$
        \item Test vectors $a$ and $b$
        \item $P(a)$ cover a list of instructions $C_a$
        \item If $C_a = C_b$ they are in the same equivalence partition
    \end{itemize}
\end{frame}

\begin{frame}
    \frametitle{Equivalence Partitioning in Testing}
    \begin{itemize}
        \item Impossible to cover all inputs
        \item Partition should be handled the same
        \item Cover each partition
    \end{itemize}
\end{frame}

%\begin{frame}
%    \frametitle{Example}
%    \textbf{Tickets}\\
%    Number of persons (\$100 per person)\\
%    Payment method (cash +25\%, credit card +\$1)\\~\\
%    % One for each payment method, plus negative number of persons
%\end{frame}

\begin{frame}
    \frametitle{Example}
    \textbf{Interest Rate}\\
    \begin{itemize}
        \item 0-5000\$ 3\%
        \item Over 5000\$ 5\%
        \item Loyalty bonus: +0.2\% per year (max 50 years)
    \end{itemize}
    % Test cases: 1$ 1 year, 9001$ 1year, 1$ 100years
\end{frame}

\begin{frame}
    \frametitle{Other Techniques}
    Boundary value analysis
    % Programmers often make mistakes at boundaries
\end{frame}


%------------------------------------------------

\begin{frame}
    \frametitle{The End}

    %\Huge{\centerline{The End}}
    \begin{quote}
        ``Testing shows the presence, not the absence of bugs.''
        \raggedleft{--- Edsger W. Dijkstra}
    \end{quote}
\end{frame}

%----------------------------------------------------------------------------------------

\end{document}

