\documentclass{beamer}

\mode<presentation> {

\usetheme{Copenhagen}
\usecolortheme{seagull}
}

\usepackage{graphicx}
\DeclareGraphicsExtensions{.pdf,.png,.jpg}
\usepackage{booktabs} % Allows the use of \toprule, \midrule and \bottomrule in tables
\usepackage{mathtools}
\usepackage{multirow}
\usepackage{color, colortbl}
\usepackage{minted}

%----------------------------------------------------------------------------------------
%	TITLE PAGE
%----------------------------------------------------------------------------------------

\title[Test]{Test}

\author{Rasmus Guldborg Pedersen}

\date{June 2015}

\begin{document}

\begin{frame}
\titlepage
\end{frame}

\begin{frame}
\frametitle{Overview}
\tableofcontents
\end{frame}



\section{Q 1.5: Use case and test cases}

\begin{frame}
    \frametitle{Use Case}
    A collection of possible scenarios between the system under discussion
    and external actors, characterized by the goal the primary actor has
    toward the system's declared responsibilities, showing how the primary
    actor's goal might be delivered or might fail.
    % Likely by Alistair Cockburn
    % Non functional requirements are not part of use cases
\end{frame}

\begin{frame}[shrink]
    \frametitle{Use Case Template}
    \begin{enumerate}
        \item Title
            % Short sentence naming the goal of the primary actor.
            % Example: Get a cup of coffee
        \item Primary Actor
            % Primary Actor is Normally the user
            % Secondary Actor Normally the system
        \item Goal in Context
            % Example: A cup of coffee
        \item Scope
            % Enterprise, system, subsystem
        \item Level
            % Summary level goal
            % User goal
            % Subfunction level goal
        \item Stakeholders and Interests
        \item Preconditions
            % The coffee machine has power
        \item Minimal Guarantees
            % The minimum guarantee in case of failure
        \item Success Guarantees
            % A cup of coffee is dispensed
        \item Trigger
            % Primary actor presses the button corresponding to the flavor
            % he/she wants.
        \item Main Success Scenario
            % A flavor is selected and the machine dispenses a cup of the
            % selected flavor.
        \item Extensions
            % Two kinds: Recovery and failure
            % Recovery: Water is too cold. Wait while the water is heated.
            % Failure: Empty for ingredients.
        \item Technology and Data Variations List
            % No GSM coverage -> the sale is logged locally
            % Different flavors of coffee is selected
    \end{enumerate}
\end{frame}

\begin{frame}
    \frametitle{Transforming to Test Cases}
    \begin{itemize}
        \item Main scenario $\times$ extensions $\times$ variations
            $\rightarrow$ test cases
            % Combining main scenario with extensions and variations result in
            % test cases.
        \item Precondition $\rightarrow$ entry criteria
    \end{itemize}
\end{frame}

\begin{frame}
    \frametitle{IEEE 829 Test Specification}
    \begin{itemize}
        \item Test design specification identifier
            % Specify the unique identifier assigned to this test design
            % specification.  Supply a reference to the associated test plan, if
            % it exists.
        \item Features to be tested
            % Identify the test items and describe the features and combinations
            % of features that are the object of this design specification.
            % Other features may be exercised, but need not be identified.  For
            % each feature or feature combination, a reference to its associated
            % requirements in the item requirement specification or design
            % description should be included.
        \item Approach refinements
            % Specify refinements to the approach described in the test plan.
            % Specify the results of any analysis that provides a rationale for
            % test case selection.  For example, one might specify conditions
            % that permit a determination of error tolerance (e.g., those
            % conditions that distinguish valid inputs from invalid inputs).
        \item Test identification
            % List the identifier and a brief description of each test case
            % associated with this design.  A particular test case may be
            % identified in more than one test design specification.  List the
            % identifier and a brief description of each procedure associated
            % with this test design specification.
        \item Feature pass/fail criteria
            % Specify the criteria to be used to determine whether the feature
            % or feature combination has passed or failed.
    \end{itemize}
\end{frame}


%------------------------------------------------

\begin{frame}
    \frametitle{The End}

    %\Huge{\centerline{The End}}
    \begin{quote}
        ``Testing shows the presence, not the absence of bugs.''
        \raggedleft{--- Edsger W. Dijkstra}
    \end{quote}
\end{frame}

%----------------------------------------------------------------------------------------

\end{document}

